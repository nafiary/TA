\chapter{KATA PENGANTAR}

\indent\indent Puji syukur Penulis panjatkan kepada Allah SWT. atas pimpinan, penyertaan, dan karunia-Nya sehingga Penulis dapat menyelesaikan Tugas Akhir yang berjudul :
\begin{center}
	\textbf{\MakeUppercase{\judul}}.
\end{center}

Penelitian Tugas Akhir ini dilakukan untuk memenuhi salah satu syarat meraih gelar Sarjana di Departemen Informatika Fakultas Teknologi Informasi dan Komunikasi Institut Teknologi Sepuluh Nopember.

Dengan selesainya Tugas Akhir ini diharapkan apa yang telah dikerjakan Penulis dapat memberikan manfaat bagi perkembangan ilmu pengetahuan terutama di bidang teknologi informasi serta bagi diri Penulis sendiri selaku peneliti.

Penulis mengucapkan terima kasih kepada semua pihak yang telah memberikan dukungan baik secara langsung maupun tidak langsung selama Penulis mengerjakan Tugas Akhir maupun selama menempuh masa studi antara lain:

\begin{itemize}
	\item Bapak, Mama, Ghina dan keluarga Penulis yang selalu memberikan perhatian, dorongan dan kasih sayang yang menjadi semangat utama bagi diri Penulis sendiri baik selama Penulis menempuh masa perkuliahan maupun pengerjaan Tugas Akhir ini.
	\item Bapak Royyana Muslim I., S.Kom., M.Kom., Ph.D. selaku Dosen Pembimbing yang telah banyak meluangkan waktu untuk memberikan ilmu, nasihat, motivasi, pandangan dan bimbingan kepada Penulis baik selama pengerjaan Tugas Akhir ini.
	\item Ibu Henning Titi C., S.Kom., M.Kom. selaku dosen pembimbing yang telah memberikan ilmu, dan masukan kepada Penulis.
	\item Seluruh tenaga pengajar dan karyawan Departemen Informatika ITS yang telah memberikan ilmu dan waktunya demi berlangsungnya kegiatan belajar mengajar di Departemen Informatika ITS.
	\item Seluruh teman angkatan 14 di Departemen Informatika ITS yang telah memberikan dukungan dan semangat kepada Penulis selama Penulis menyelesaikan Tugas Akhir ini.
	\item Teman-teman, kakak-kakak dan adik-adik \textit{administrator} Workshop Pemrograman 2 yang selalu menjadi teman untuk berbagi ilmu.
\end{itemize}

Penulis mohon maaf apabila masih ada kekurangan pada Tugas Akhir ini. Penulis juga mengharapkan kritik dan saran yang membangun untuk pembelajaran dan perbaikan di kemudian hari. Semoga melalui Tugas Akhir ini Penulis dapat memberikan kontribusi dan manfaat yang sebaik-baiknya. \\ \\ \\

\hfill Surabaya, Mei \tahun \\ \\ \\

\hfill \penulis \\
\cleardoublepage
